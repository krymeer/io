\chapter{Wstęp}
\thispagestyle{chapterBeginStyle}

Praca poświęcona jest modułowi \textbf{CSS Grid Layout}. Technologia ta, rozwijana od 2011 roku przez World Wide Web Consortium (\textbf{W3C}), jest odpowiedzią na ciągle aktualną potrzebę budowania nowoczesnych stron WWW. CSS Grid Layout pozwala na używanie na stronach internetowych \textbf{gridów} -- prostokątnych elementów, których obszar można dzielić na mniejsze fragmenty, zarówno w pionie, jak i w poziomie. Autorzy CSS Grid Layout przygotowali ponadto odpowiednie reguły umożliwiające uzyskanie pożądanego wyglądu gridów i bloków składających się na nie. Innowacyjność modułu wynika z tego, że pozwala on na dzielenie widoku stron WWW w\,dwóch wymiarach, co do tej pory wiązało się z rozmaitymi problemami i niedogodnościami.

Podstawowym celem pracy było zapoznanie się z modułem CSS Grid Layout i znalezienie jego mocnych oraz słabych stron; kolejnym krokiem stało się przygotowanie szablonów stron WWW i napisanie generatora umożliwiającego stworzenie prostej strony internetowej. W wyniku prac nad projektem powstała aplikacja webowa -- działająca na urządzeniach o różnych rozdzielczościach ekranu -- przy użyciu której można zaprojektować szablon strony WWW, podzielić go na bloki, wypełnić tekstem, ostylować wybrane elementy i\,wygenerować gotowy plik HTML. Co istotne, z aplikacji można korzystać, nie mając dostępu do Internetu. Projekt nie wymaga także użycia serwera WWW.

W pracy wykorzystano nowoczesne technologie z zakresu budowania stron WWW -- przy definiowaniu wyglądu aplikacji i generowanych przez nią szablonów, oprócz wspomnianego CSS Grid Layout, swoje zastosowanie znalazły języki \textbf{HTML} i \textbf{CSS} w najwyższych wersjach -- odpowiednio piątej i trzeciej. Za poprawną interakcję między aplikacją a użytkownikiem odpowiada kod interpretowany przez bibliotekę \textbf{jQuery}, operującą na języku \textbf{JavaScript}. Nieoceniony okazał się także edytor tekstu \textbf{Sublime Text}, ułatwiający pisanie prawidłowego kodu. Kolejne wersje projektu były utrwalane i aktualizowane przy pomocy repozytorium działającego na systemie kontroli wersji \textbf{Git}.

Praca dyplomowa składa się z sześciu rozdziałów.

W rozdziale drugim omówiono podstawowe cechy modułu CSS Grid Layout i dostarczane przezeń rozwiązania wraz z przykładami pokazującymi w praktyce, jak można wykorzystać tę technologię. Rozdział trzeci skupia się się na problemach związanych z CSS Grid Layout -- zbadane zostało to, jakie oprogramowanie wspiera ten moduł i które jego cechy wymagają dopracowania. W rozdziale czwartym przedstawiono aplikację stworzoną w ramach pracy dyplomowej -- generator szablonów stron WWW -- oraz jej dokumentację techniczną i wykorzystywane technologie. Rozdział piąty przybliża instalację aplikacji w systemie operacyjnym; opisuje też, co należy zrobić, by stworzyć w niej własny szablon strony internetowej. W rozdziale szóstym zamieszczono podsumowanie pracy i realizowanego projektu.